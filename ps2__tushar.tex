%% LyX 2.3.2-2 created this file.  For more info, see http://www.lyx.org/.
%% Do not edit unless you really know what you are doing.
\documentclass[11pt]{article}
\usepackage[latin9]{inputenc}
\usepackage{geometry}
\geometry{verbose}
\usepackage{textcomp}
\usepackage{amsmath}
\usepackage{amssymb}

\makeatletter

%%%%%%%%%%%%%%%%%%%%%%%%%%%%%% LyX specific LaTeX commands.
\InputIfFileExists{t2aenc.def}{}{%
  \errmessage{File `t2aenc.def' not found: Cyrillic script not supported}}
\DeclareRobustCommand{\cyrtext}{%
  \fontencoding{T2A}\selectfont\def\encodingdefault{T2A}}
\DeclareRobustCommand{\textcyr}[1]{\leavevmode{\cyrtext #1}}


%%%%%%%%%%%%%%%%%%%%%%%%%%%%%% User specified LaTeX commands.
% who=0 student solutions
% who=1 - my solutions
% who=2 - assignment
% who=3 recitation problems
% who=4 recitation solution sketches
% Everyone changes these:
\def\who{2}                        %Set to 0 when filling solutions
\def\name{Fill Submitter's Name}          %Student name
% Instructors will set these:
\def\num{2}  %homework number
\def\assigneddate{September 20, 2019}  %assigned date/start date for recitations
\def\duedate{September 27, 2019}  %due date/end date for recitations
\def\course{CS5800 Algorithms} %course name
\def\instructorname{Ravi Sundaram}          %Instructor's name
%
%

% Don't change anything below
\usepackage{amsfonts}
\usepackage{latexsym}


\ifnum\who<3
\renewcommand{\thepage}{Problem Set \num, Page \arabic{page}}
\fi
\ifnum\who>2
\renewcommand{\thepage}{Recitation (\assigneddate{}~-~\duedate{}), Page \arabic{page}}
\fi

% \setlength{\fboxsep}{1.5ex}
% \setlength{\fboxrule}{0.25ex}
% Arguments are in order: assigned date, Name, due date and central text
\newcommand{\handout}[4]{
  \noindent
  \begin{center}
    \framebox{
      \parbox{0.9\textwidth}{
        \begin{minipage}[c][1.5em][t]{0.9\textwidth}
          \textbf{\course} \hfill \textit{#1}
          \end{minipage}
        \begin{minipage}[c][2em][c]{0.9\textwidth}
          {\Large \hfill #4  \hfill}
          \end{minipage}
        \begin{minipage}[c][1.5em][b]{0.9\textwidth}
          \textbf{#2} \hfill \textit{#3}
          \end{minipage}
      }
    }
  \end{center}
  \vspace{2ex}
}

\newcounter{problemCounter}
\newcommand{\problemNumber}{\arabic{problemCounter}}  %problem number
\newcommand{\increaseProblemCounter}{\addtocounter{problemCounter}{1}}  %problem number

%first argument desription, second number of points
\newcommand{\newproblem}[2]{
\increaseProblemCounter
\section*{Problem \problemNumber~(#1) \hfill {#2}}
}

\newcommand{\qed}{\newline\hspace*{\fill}$\blacksquare$}
\newenvironment{solution}{\par\noindent\subsection*{Solution:}}{\qed}
\newenvironment{solsketch}{\par\noindent\subsection*{Solution Sketch:}}{\qed}

\renewcommand{\eqref}[1]{Equation~(\ref{eq:#1})}

\newcommand{\hint}[1]{(\textbf{Hint}: {#1})}
%Put more macros here, as needed.
\newcommand{\room}{\medskip\ni}
\newcommand{\set}[1]{\{#1\}}

\makeatother

\begin{document}
% Choosing the right arguments to \handout, don't edit here, edit variables above
\ifnum\who=0 \handout{Out: \assigneddate}{Name: \name{}}{Due:
\duedate}{Solutions to Problem Set \num} \textbf{Group members:}
{*}{*}{*}{*} Write the names of all group members here. Use the submitter's
name in the \verb|\name{}| command. {*}{*}{*}{*} \fi \ifnum\who=1
\handout{Out: \assigneddate}{Name: \name{}}{Due: \duedate}{Solutions
for Problem Set \num} \fi \ifnum\who=2 \handout{Out: \assigneddate}{\instructorname}{Due:
\duedate}{Problem Set \num} \fi \ifnum\who=3 \handout{}{\instructorname}{}{Recitations
from \assigneddate{} to \duedate} \fi \ifnum\who=4 \handout{}{\instructorname}{}{Solutions
for Recitations from \assigneddate{} to \duedate} \fi

\newproblem{Solving recurrences}{10+5} Solve the recurrence $T(n)=2T(\sqrt{n})+1$
first by directly adding up the work done in each iteration and then
using the Master Theorem.

\newproblem{Recursion}{5+10} Consider the following recursive function
$\ell(n)$:

\begin{align*}
\ell(n)=\begin{cases}
0 & \text{ if }n=0\\
\ell(n/2) & \text{ if n is even}\\
1+\ell((n-1)/2) & \text{ if n is odd}
\end{cases}
\end{align*}

\begin{enumerate}
\item Describe what the function $\ell(n)$ is calculating. Your description
should make sense to a computer scientist, but be independent of the
recursive definition. 
\item Give a recurrence relation for the run time of $\ell(n)$, and solve
it to get the run time of $\ell(n)$.
\end{enumerate}

\section*{Solution:}

\subsection*{To write a recurrence relationship, first we will need to make a
few assumptions:}
\begin{enumerate}
\item $l((n-1)/2)$ takes almost as much time as $l(n/2)$ in the worst
case
\item We will have to add 1 for every step in the worst case
\end{enumerate}

\subsection*{Given above 2 assumptions we can say that, T(n) will be to add a
constant and a recurrence of T(n/2).}

\subsection*{Therefore we get the following recurrence relationship:}

\[
T(n)=T(\frac{n}{2})+1
\]


\subsection*{Using master theorem we get: a = 1, b = 2, f(n) = 1, log$_{b}$a
= 0}

\subsection*{We can write f(n) as log$^{k}$n where k = 0. Hence this becomes
the case 2 of master theorem.}

\subsection*{Case 2 of master theorem suggests run time of O($n^{log_{b}a}.log^{k+1}n$)
where k\textgreater =0}

\subsection*{Substituting, log$_{b}$a = 0 and k = 0}

\subsection*{We get T(n) = \textcyr{\CYROTLD}(n$^{0}.log^{0+1}n$) = \textcyr{\CYROTLD}(log(n)).}

\newproblem{Sorting special arrays}{3+4+10+3} Consider the problem
of sorting an array $A[1,...,n]$ of integers. We presented an $O(n\log n)$-time
algorithm in class and, also, proved a lower bound of $\Omega(n\log n)$
for any comparison-based algorithm. 
\begin{enumerate}
\item Give an efficient sorting algorithm for a \textbf{\em boolean}\footnote{In a boolean array $B[1,...,n]$, each element $B[i]$ (for $i=1,...,n$)
is either $0$ or $1$.} array $B[1,...,n]$.
\item Give an efficient sorting algorithm for an array $C[1,...,n]$ whose
elements are taken from the set $\{1,2,3,4,5\}$.
\item Give an efficient sorting algorithm for an array $D[1,...,n]$ whose
elements are distinct ($D[i]\not=D[j]$, for every $i\not=j\in\{1,...,n\}$)
and are taken from the set $\{1,2,...,2n\}$.
\item In case you designed linear-time sorting algorithms for the previous
subparts, does it mean that the lower bound for sorting of $\Omega(n\log n)$
is wrong? Explain.
\end{enumerate}
\newpage{}

\section*{Solution:}

\subsection*{1. Boolean Array of size 'n' can be sorted in O(n) i.e. linear time
using following counting algorithm:}

i,count0=0

while i \textless{} n:

	if B{[}i{]}==0 then count0++	//O(n) work done

	i++

newB{[}0:count0{]} = array of 0's

newB{[}count0:n{]} = array of 1's

\subsection*{2. Array C with 'n' elements from set \{1,2,3,4,5\} of 'k=5' items
can be done using the same counting algorithm. This time we will have
to do at worse 'k' work for every element in array C.}

i=0,countArray={[}0{]}{*}5

for i in C:

	countArray{[}i-1{]}++	

	i++

\newpage{}

\newproblem{Lower Bounds}{10} Show a $\Omega(\log n)$ lower bound
for finding elements in a sorted array.

\section*{Solution:}

\subsection*{If we have a list with 'n' elements, we can make a binary tree with
total 'n' leaves. Then we will have to make 1 comparison at every
level of the tree.}

\subsection*{Assuming 'h' to be the total number of levels or height of the tree
we get the following:}

\subsection*{Maximum number of leaves = $2^{h}$ and we can also say, T(n) will
be O(h) since there is 1 comparison at every level.}

\subsection*{Using the idea of comparison trees for lower bounds, we can get the
relationship between total number of leaves and minimum number of
leaves we need.}

\subsection*{Therefore,}

\[
2^{h}>total_{-}leaves>n
\]


\subsection*{Taking log$_{2}$ on each side,}

\[
h.log_{2}2>log_{2}n
\]

\[
h>log_{2}n
\]


\subsection*{Looking at above relationship we can say that height of the tree
which is also the total work done T(n) is lower bounded by log$_{2}$n.}

\subsection*{Therefore, for selection in a sorted array T(n) is $\text{\textohm(log n)}$.}

\newproblem{Programming: Counting threshold inversions}{40}

You'll be given an array (of integers) and a threshold value as input,
write a program to return the number of threshold inversions in the
array. An inversion between indices $i<j$ is a threshold inversion
if $a_{i}>t*a_{j}$, where $t$ is the threshold value given as input.

\emph{Hint: Understanding counting inversions (normal inversions,
without a threshold) will be useful to figure out how to do threshold
inversions. We will be covering that in the recitations.} {document}
}{}{}}{}{}
\end{document}
