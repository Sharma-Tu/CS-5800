%% LyX 2.3.2-2 created this file.  For more info, see http://www.lyx.org/.
%% Do not edit unless you really know what you are doing.
\documentclass[11pt]{article}
\usepackage[latin9]{inputenc}
\usepackage{geometry}
\geometry{verbose}
\usepackage{amsmath}
\usepackage{amssymb}

\makeatletter
%%%%%%%%%%%%%%%%%%%%%%%%%%%%%% User specified LaTeX commands.
% who=0 student solutions
% who=1 - my solutions
% who=2 - assignment
% who=3 recitation problems
% who=4 recitation solution sketches
% Everyone changes these:
\def\who{2}                        %Set to 0 when filling solutions
\def\name{Fill Submitter's Name}          %Student name
% Instructors will set these:
\def\num{1}  %homework number
\def\assigneddate{September 13, 2019}  %assigned date/start date for recitations
\def\duedate{September 20, 2019}  %due date/end date for recitations
\def\course{CS5800 Algorithms} %course name
\def\instructorname{Ravi Sundaram}          %Instructor's name
%
%

% Don't change anything below
\usepackage{amsfonts}
\usepackage{latexsym}\usepackage{algorithm}
\usepackage{algpseudocode}


\ifnum\who<3
\renewcommand{\thepage}{Problem Set \num, Page \arabic{page}}
\fi
\ifnum\who>2
\renewcommand{\thepage}{Recitation (\assigneddate{}~-~\duedate{}), Page \arabic{page}}
\fi

% \setlength{\fboxsep}{1.5ex}
% \setlength{\fboxrule}{0.25ex}
% Arguments are in order: assigned date, Name, due date and central text
\newcommand{\handout}[4]{
  \noindent
  \begin{center}
    \framebox{
      \parbox{0.9\textwidth}{
        \begin{minipage}[c][1.5em][t]{0.9\textwidth}
          \textbf{\course} \hfill \textit{#1}
          \end{minipage}
        \begin{minipage}[c][2em][c]{0.9\textwidth}
          {\Large \hfill #4  \hfill}
          \end{minipage}
        \begin{minipage}[c][1.5em][b]{0.9\textwidth}
          \textbf{#2} \hfill \textit{#3}
          \end{minipage}
      }
    }
  \end{center}
  \vspace{2ex}
}

\newcounter{problemCounter}
\newcommand{\problemNumber}{\arabic{problemCounter}}  %problem number
\newcommand{\increaseProblemCounter}{\addtocounter{problemCounter}{1}}  %problem number

%first argument desription, second number of points
\newcommand{\newproblem}[2]{
\increaseProblemCounter
\section*{Problem \problemNumber~(#1) \hfill {#2}}
}

\newcommand{\qed}{\hspace*{\fill}$\blacksquare$}
\newenvironment{solution}{\par\noindent\subsection*{Solution:}}{\qed}
\newenvironment{solsketch}{\par\noindent\subsection*{Solution Sketch:}}{\qed}

\renewcommand{\eqref}[1]{Equation~(\ref{eq:#1})}

\newcommand{\hint}[1]{(\textbf{Hint}: {#1})}
%Put more macros here, as needed.
\newcommand{\room}{\medskip\ni}
\newcommand{\set}[1]{\{#1\}}



\makeatother

\begin{document}
% Choosing the right arguments to \handout, don't edit here, edit variables above
\ifnum\who=0 \handout{Out: \assigneddate}{Name: \name{}}{Due:
\duedate}{Solutions to Problem Set \num} \textbf{Group members:}
{*}{*}{*}{*} Write the names of all group members here. Use the submitter's
name in the \verb|\name{}| command. {*}{*}{*}{*} \fi \ifnum\who=1
\handout{Out: \assigneddate}{Name: \name{}}{Due: \duedate}{Solutions
for Problem Set \num} \fi \ifnum\who=2 \handout{Out: \assigneddate}{\instructorname}{Due:
\duedate}{Problem Set \num} \fi \ifnum\who=3 \handout{}{\instructorname}{}{Recitations
from \assigneddate{} to \duedate} \fi \ifnum\who=4 \handout{}{\instructorname}{}{Solutions
for Recitations from \assigneddate{} to \duedate} \fi

\newproblem{Proof by Contradiction}{10 points}

$P$ is a given program which takes an input string and always returns
1. (That is, program $P$ computes the function $f(n)=1$, for every
$n\in\mathbb{N}=\{1,2,...\}$.) Prove that there is no equality checking
program which when given another program $Q$ on input, can conclusively
determine whether $P=Q$ (i.e. in terms of behavior $Q$ computes
the same function as $P$).

\newproblem{Proof by Induction}{20 points}
\begin{enumerate}
\item For all $n\geq1$ prove that the following summation formula is correct:
\begin{align*}
\frac{1}{1\times3}+\frac{1}{3\times5}+\frac{1}{5\times7}+\cdots+\frac{1}{(2n-1)(2n+1)}=\frac{n}{2n+1}
\end{align*}
\item Prove that $2^{n}+1$ is divisible by 3 for all odd natural numbers
$n$. 
\item Let $n$ and $k$ be non-negative integers with $n\geq k$. Prove
$\sum_{i=k}^{n}{i \choose k}={n+1 \choose k+1}$.
\end{enumerate}

\paragraph*{Solution:}

Base Case for n=k:

L.H.S = $\mathop{\sum_{i=k}^{k}\binom{i}{k}}=\binom{k}{k}=1$

R.H.S = $\mathop{\mathop{\binom{k+1}{k+1}}=1}$

Inductive step for n = k+m:

L.H.S = $\mathop{\sum_{i=k}^{k+m}\binom{i}{k}}=\binom{k}{k}+\binom{k+1}{k}+.......+\binom{k+m}{k}$

R.H..S = $\mathop{\binom{k+m+1}{k+1}}$

We need to prove the given statement for n = k+m+1:

which is to prove: $\mathop{\sum_{i=k}^{k+m+1}\binom{i}{k}=\mathop{\binom{k+m+1+1}{k+1}}}$

L.H.S $=\binom{k}{k}+\binom{k+1}{k}+.......+\binom{k+m}{k}+\binom{k+m+1}{k}$

We have first 'm' terms same as L.H.S of inductive step, so we can
replace the series with R.H.S:

$=[\binom{k}{k}+\binom{k+1}{k}+.......+\binom{k+m}{k}]+\binom{k+m+1}{k}=\binom{k+m+1}{k+1}+\binom{k+m+1}{k}$

R.H.S =$\mathop{\binom{(k+m+1)+1}{k+1}}$

We know from Pascal's Rule: $\mathop{\mathop{\binom{n+1}{k}}=\binom{n}{k}+\binom{n}{k-1}}$

Applying above rule for n = k+m+1 in R.H.S we get,

R.H.S =$\mathop{\binom{k+m+1}{k+1}+\binom{k+m+1}{k}}$

This is the same as L.H.S and hence the given statement holds true.
\begin{enumerate}
\item Let $f_{0}=0,f_{1}=f_{2}=1$ and define $f_{n}=f_{n-1}+f_{n-2}$ for
all $n\geq1$. These are called the Fibonacci numbers. Prove that
for all $n\geq1$, the following holds: 
\begin{align*}
f_{n+1}<\left(\frac{7}{4}\right)^{n}
\end{align*}
\end{enumerate}
\newproblem{Asymptotics}{20 points} Arrange the following functions
in ascending order of growth rate. Also give reasoning for the order.
\begin{align*}
g_{01}(n) & =n^{\frac{101}{100}}\hskip0.2\textwidth &  & g_{02}(n)=n2^{n+1}\\
g_{03}(n) & =n(\log n)^{3}\hskip0.2\textwidth &  & g_{04}(n)=n^{\log\log n}\\
g_{05}(n) & =\log(n^{2n})\hskip0.2\textwidth &  & g_{06}(n)=n!\\
g_{07}(n) & =2^{\sqrt{\log n}}\hskip0.2\textwidth &  & g_{08}(n)=2^{2^{n+1}}\\
g_{09}(n) & =\log(n!)\hskip0.2\textwidth &  & g_{10}(n)=\lceil\log(n)\rceil!\\
g_{11}(n) & =2^{\log\sqrt{n}}\hskip0.2\textwidth &  & g_{12}(n)=\sqrt{2}^{\log n}
\end{align*}

\textbf{Solution:}

\newproblem{More Asymptotics}{20 points}

For a given two functions $f$, $g$ and $h$. Decide whether each
of the following statements are correct and give a proof for each
part. 
\begin{enumerate}
\item If $f(n)=\Omega(g(n))$ and $g(n)=\Omega(f(n))$, then $f(n)=\Theta(g(n))$ 
\item If $f(n)=o(g(n))$, then $g(n)\not\in O(f(n))$ 
\item If $f(n)=O(h(n))$ and $g(n)=O(h(n))$, then $f(n)+g(n)=O(h(n))$ 
\item If $f(n)=O(h(n))$ and $g(n)=O(h(n))$, then $f(n)\cdot g(n)=O(h(n))$ 
\item If $f(n)=O(g(n))$, then $2^{f(n)}=O(2^{g(n)})$
\end{enumerate}

\paragraph{Solution:}

If $f(n)=O(g(n))$ then $\mathop{f(n)\le c_{1}.g(n)}$ where $c_{c}>0$

Since $\mathop{f(n)\le c_{1}.g(n)}$ then if we raise both sides to
the power of 2 then

$2^{f(n)}\text{\ensuremath{\le}}2^{c_{1}.g(n)}$

$2^{f(n)}\text{\ensuremath{\le}}2^{c_{1}}.2^{g(n)}$

Let $2^{c_{1}}=c_{2}.$Since $c_{1}>0$ and $c_{2}\text{\ensuremath{\ge}}1$

Thus, $2^{f(n)}\text{\ensuremath{\le}}c_{2}.2^{g(n)}$

therefore, $2^{f(n)}=O(2^{g(n)})$

% \newpage

\newproblem{Programming: Modular exponentiation}{30 points}

See Piazza for a post containing a link to the HackerRank page, where
you will be submitting the assignment. In addition to the below description,
it also contains more formal requirements for how your program should
behave.

\noindent \textbf{Description:} Implement modular exponentiation in
a way that outputs the intermediary steps of the algorithm.

\noindent \textbf{Problem Statement:} {*}{*}IMPORTANT{*}{*} You are
NOT allowed to use built in language functions which trivialize the
task of computing exponents. Any submissions which use this and avoid
the task at hand will be given a 0.

In this problem, you will have to efficiently implement modular exponentiation.
Recall that the problem of modular exponentiation is, given positive
integers $a$ and $n$, and a non-negative integer $x$, calculate
$a^{x}\mod n$.

One way of doing this is exponentiation by squaring. It involves repeatedly
squaring the base $a$ and reducing it $\mod n$. Doing so yields
the values $a$, $a^{2}\mod n$, $a^{4}\mod n$, $a^{8}\mod n$, ...
etc. By combining these in the correct way, and using the fact that
every number has a binary representation, we can compute $a^{x}\mod n$
in time $O(\log x)$. Implement modular exponentiation by squaring,
and output the intermediary values of $a^{2^{i}}\mod n$, as well
as the final value $a^{x}\mod n$.

Note that in addition to the above, the challenge page also describes
the input/output format, and gives a few examples.

{document} }{}{}}{}{}
\end{document}
